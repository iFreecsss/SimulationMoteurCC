
%------------------------------------------------------------------------------%
\documentclass[a4paper,12pt]{article}
\usepackage[utf8]{inputenc}
\usepackage{graphicx}
\usepackage{amsmath}
\usepackage[french]{babel}
\usepackage{comment}
\usepackage{amssymb}
\usepackage{float}
\usepackage{tikz}
\usepackage{booktabs}
\usetikzlibrary{shapes.geometric} % Charger les formes géométriques
\usepackage{setspace}
\usepackage{subcaption}
\usepackage{geometry}
\usepackage[utf8]{inputenc}
\usepackage{hyperref}
\usepackage{caption}
\usepackage{subcaption}
\usepackage{enumitem}
\usepackage{afterpage}
\usepackage{algorithm2e}
\makeatletter
\newlength{\boxed@align@width}
\newcommand{\boxedalign}[2]{
	#1 & \setlength{\boxed@align@width}{\widthof{$\displaystyle#1$}+0.1389em+\fboxsep+\fboxrule}
	\hspace{-\boxed@align@width}\addtolength{\boxed@align@width}{-\fboxsep-\fboxrule}\boxed{\vphantom{#1}\hspace{\boxed@align@width}#2}}
\makeatother
\usetikzlibrary{shapes.geometric, arrows}
\captionsetup{
	justification=centering, %cente la légende
}
\geometry{left=2cm, right=2cm, top=2cm, bottom=2cm}
\renewcommand{\contentsname}{Sommaire}
\renewcommand{\refname}{Bibliographie}


\begin{document}
	
	% Title Page
	\begin{titlepage}
		\begin{flushleft}{\Large \includegraphics[width=0.2\textwidth]{Logo_SU.svg.png}}\hfill \raisebox{0.4cm}{\textbf{UM4RBR14}}
		\end{flushleft}
		
		
		\vspace{1cm} % Adjust space to move the title to the center of the page
		
		\vfill
		\begin{center}
			
		\end{center}
		
		\vfill
		\begin{center}
			\rule{\textwidth}{0.5mm} \\[1.2cm]
			{\LARGE \textbf{Compte rendu de Simulation}}\\[1cm]
			\rule{\textwidth}{0.5mm} \\[2.5cm]
		\end{center}
		\vfill
		
		
		
		\vfill
		\begin{center}
			
		\end{center}
		
		
		\begin{flushleft}
			\textbf{Paul CAMUS} \\
			\textbf{21107953 M1 ROB}\hfill \textbf{\today}
		\end{flushleft}
		
		
	\end{titlepage}
	\onehalfspacing
	\hypersetup{linkcolor=black}
	\setcounter{tocdepth}{3}
	\newgeometry{margin=1.5in}
	
	\section*{Introduction}
	
	
		\section{Simulation d'un moteur à courant continu}
		
		\subsection{Solution analytique}
		Rappelons dans un premier temps les relations électriques et mécaniques à partir desquelles se construit le modèle.
		
		\begin{enumerate}
			\item \textbf{Équation électrique}:
			\begin{equation}
				U_m(t) = E(t) + R \cdot i(t) + L \frac{di(t)}{dt}
				\label{eq_elec}
			\end{equation}
			
			\item \textbf{Équation mécanique}:
			\begin{equation}
				J \frac{d\Omega(t)}{dt} + f \cdot \Omega(t) = \Gamma(t)
				\label{eq_meca}
			\end{equation}
			
			\item \textbf{Équations de couplage}:
			\begin{equation}
				\begin{cases}
					\Gamma(t) = k_c \cdot i(t) \\
					E(t) = k_e \cdot \Omega(t)
				\end{cases}
				\label{eq_couplage}
			\end{equation}
		\end{enumerate}
		
		Par ailleurs, on suppose que l'inductance est négligeable ()$L \approx 0$).L'\autoref{eq_elec} devient :
		\begin{equation}
			U_m(t) = E(t) + R \cdot i(t) \implies i(t) = \frac{U_m(t) - E(t)}{R}
		\end{equation}
		
		En injectant l'expression de $i(t)$ obtenue dans l'\autoref{eq_meca} on peut développer :
		
		\begin{align}
			&J \frac{d\Omega(t)}{dt} + f \cdot \Omega(t) = k_c \left( \frac{U_m(t) - k_e \cdot \Omega(t)}{R} \right)\notag\\
			&J \frac{d\Omega(t)}{dt} + f \cdot \Omega(t) = \frac{k_c}{R} U_m(t) - \frac{k_c k_e}{R} \Omega(t)\notag\\
			&J \frac{d\Omega(t)}{dt} + \left( f + \frac{k_c k_e}{R} \right) \Omega(t) = \frac{k_c}{R} U_m(t)\notag\\
			&\boxed{\tau \frac{d\Omega(t)}{dt} + \Omega(t) = K \cdot U_m(t)}\label{forme_cano}
		\end{align}
		
		On remarque qu'il s'agit de la forme canonique du premier ordre, on peut alors identifier le gain statique $K$ et la constante de temps $\tau$ :
		\begin{equation}
		\begin{cases}
			K = \frac{k_c}{R f + k_c k_e}\\
			\tau = \frac{R J}{R f + k_c k_e}
		\end{cases}
		\label{syst_Ktau}
		\end{equation}
		
		Enfin, pour trouver la réponse indicielle, on considère une entrée échelon $U_m(t) = U_0$ pour $t \ge 0$ et avec des conditions initiales nulles $\Omega(0)=0$. En combinant le système \ref{syst_Ktau} et l'\autoref{forme_cano}
		
		\begin{align}
			&\Omega(t) = K \cdot U_0 \left( 1 - e^{-t/\tau} \right)\notag\\
			&\boxed{ \Omega(t) = \frac{k_c U_0}{R f + k_c k_e} \left( 1 - \exp\left( \frac{-(R f + k_c k_e)t}{R J} \right) \right)}\label{sol_analytique}
		\end{align}
		
		\subsection{Simulation}
\end{document}