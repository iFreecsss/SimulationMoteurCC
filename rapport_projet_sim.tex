
%------------------------------------------------------------------------------%
\documentclass[a4paper,12pt]{article}
\usepackage[utf8]{inputenc}
\usepackage{graphicx}
\usepackage{amsmath}
\usepackage[french]{babel}
\usepackage{comment}
\usepackage{amssymb}
\usepackage{float}
\usepackage{tikz}
\usepackage{booktabs}
\usetikzlibrary{shapes.geometric} % Charger les formes géométriques
\usepackage{setspace}
\usepackage{subcaption}
\usepackage{geometry}
\usepackage[utf8]{inputenc}
\usepackage{hyperref}
\usepackage{caption}
\usepackage{subcaption}
\usepackage{enumitem}
\usepackage{afterpage}
\usepackage{algorithm2e}
\usetikzlibrary{shapes.geometric, arrows}
\captionsetup{
	justification=centering, %cente la légende
}
\geometry{left=2cm, right=2cm, top=2cm, bottom=2cm}
\renewcommand{\contentsname}{Sommaire}
\renewcommand{\refname}{Bibliographie}


\begin{document}
	
	% Title Page
	\begin{titlepage}
		\begin{flushleft}{\Large \includegraphics[width=0.2\textwidth]{Logo_SU.svg.png}}\hfill \raisebox{0.4cm}{\textbf{UM4RBR14}}
		\end{flushleft}
		
		
		\vspace{1cm} % Adjust space to move the title to the center of the page
		
		\vfill
		\begin{center}
			
		\end{center}
		
		\vfill
		\begin{center}
			\rule{\textwidth}{0.5mm} \\[1.2cm]
			{\LARGE \textbf{Compte rendu de Simulation}}\\[1cm]
			\rule{\textwidth}{0.5mm} \\[2.5cm]
		\end{center}
		\vfill
		
		
		
		\vfill
		\begin{center}
			
		\end{center}
		
		
		\begin{flushleft}
			\textbf{Paul CAMUS} \\
			\textbf{21107953 M1 ROB}\hfill \textbf{\today}
		\end{flushleft}
		
		
	\end{titlepage}
	\onehalfspacing
	\hypersetup{linkcolor=black}
	\setcounter{tocdepth}{3}
	\newgeometry{margin=1.5in}
	
	\section*{Introduction}
	
	
		\section*{Solution analytique de $\Omega(t)$ en fonction de $U_m(t)$}
		
		\subsection*{1. Équations du système}
		D'après le sujet du projet, le moteur à courant continu est régi par les équations suivantes :
		
		\begin{itemize}
			\item \textbf{Équation électrique} (loi des mailles) :
			\begin{equation}
				U_m(t) = E(t) + R \cdot i(t) + L \frac{di(t)}{dt}
			\end{equation}
			
			\item \textbf{Équation mécanique} (principe fondamental de la dynamique) :
			\begin{equation}
				J \frac{d\Omega(t)}{dt} + f \cdot \Omega(t) = \Gamma(t)
			\end{equation}
			
			\item \textbf{Equations de couplage} :
			\begin{equation}
				\Gamma(t) = k_c \cdot i(t) \quad \text{et} \quad E(t) = k_e \cdot \Omega(t)
			\end{equation}
		\end{itemize}
		
		\subsection*{2. Hypothèse simplificatrice}
		On suppose l'inductance de l'induit négligeable : $L \approx 0$.
		L'équation électrique devient alors purement algébrique :
		\begin{equation}
			U_m(t) = E(t) + R \cdot i(t) \implies i(t) = \frac{U_m(t) - E(t)}{R}
		\end{equation}
		
		\subsection*{3. Démonstration}
		En substituant l'expression du courant $i(t)$ dans l'équation mécanique, nous obtenons :
		
		\begin{equation}
			J \frac{d\Omega(t)}{dt} + f \cdot \Omega(t) = k_c \left( \frac{U_m(t) - k_e \cdot \Omega(t)}{R} \right)
		\end{equation}
		
		Développons pour regrouper les termes en $\Omega(t)$ :
		\begin{equation}
			J \frac{d\Omega(t)}{dt} + f \cdot \Omega(t) = \frac{k_c}{R} U_m(t) - \frac{k_c k_e}{R} \Omega(t)
		\end{equation}
		
		\begin{equation}
			J \frac{d\Omega(t)}{dt} + \left( f + \frac{k_c k_e}{R} \right) \Omega(t) = \frac{k_c}{R} U_m(t)
		\end{equation}
		
		En divisant par le coefficient devant $\Omega(t)$, soit $\left( f + \frac{k_c k_e}{R} \right)$, on obtient la forme canonique du premier ordre :
		\begin{equation}
			\tau \frac{d\Omega(t)}{dt} + \Omega(t) = K \cdot U_m(t)
		\end{equation}
		
		Avec les constantes identifiées :
		\begin{itemize}
			\item \textbf{Gain statique} : $K = \frac{k_c}{R f + k_c k_e}$
			\item \textbf{Constante de temps} : $\tau = \frac{R J}{R f + k_c k_e}$
		\end{itemize}
		
		\subsection*{4. Solution temporelle (Réponse indicielle)}
		Pour une entrée en échelon de tension $U_m(t) = U_0$ pour $t \ge 0$ (avec conditions initiales nulles $\Omega(0)=0$), la solution de l'équation différentielle est :
		
		\begin{equation}
			\Omega(t) = K \cdot U_0 \left( 1 - e^{-t/\tau} \right)
		\end{equation}
		
		Soit en remplaçant $K$ et $\tau$ par leurs expressions :
		\begin{equation}
			\boxed{ \Omega(t) = \frac{k_c U_0}{R f + k_c k_e} \left( 1 - \exp\left( \frac{-(R f + k_c k_e)t}{R J} \right) \right) }
		\end{equation}
		
\end{document}